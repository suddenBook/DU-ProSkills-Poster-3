\documentclass[final]{beamer}
\setbeamertemplate{navigation symbols}{}
\usepackage[size=a0,scale=1.55]{beamerposter}
\usetheme{Berlin}

\usepackage{graphicx} 

\usepackage{pgfplotstable} 
\usepackage{pgfplots}
\pgfplotsset{compat=1.17}

\usepackage{caption}

\settowidth{\leftmargini}{\usebeamertemplate{itemize item}}
\addtolength{\leftmargini}{2\labelsep}

\usepackage[utf8]{inputenc}
\usepackage[T1]{fontenc}
\usepackage[scale=0.8]{newpxtext} 

\usepackage{tikz}
\usetikzlibrary{shapes, arrows.meta, positioning}

\title{Statistics: Unlocking the Power of Data}
\author{Group 9}
\institute{Target audiences: A-Level students}
\date{\today}

\begin{document}

\begin{frame}[t]{}

	\vspace{-0.6cm}

	\begin{columns}[t]
		
		% Column 1
		\begin{column}{.3\textwidth}

			\begin{block}{What is Statistics?}

				Statistics is the science of learning from data. It enables us to understand patterns and make decisions based on data analysis.

				\vspace{1cm}

				For example, by examining the variability in A-Level mathematics results between summer 2020 and summer 2021, we can gain insights into how the pandemic might have affected student performance and assessment methods. This understanding can then inform educational strategies and policies.

				\vspace{1cm}

				\begin{figure}
					
					\begin{tikzpicture}

						\begin{axis}[
								name=myaxis,
								ybar,
								bar width=10pt,
								width=20cm,
								height=15cm,
								xlabel={Change in centre percentage of grade A or above},
								ylabel={Number of centres},
								xlabel style={font=\footnotesize},
								ylabel style={font=\footnotesize},
								tick label style={font=\tiny},
								legend style={font=\tiny},
								xmin=-50, xmax=50,
								ymin=0, ymax=170,
								xtick={-50,-40,...,50},
								ytick={0,50,...,200},
								legend pos=north east,
								ymajorgrids=true,
								grid style=dashed,
							]

							\addplot[
								fill=purple,
								draw=black,
							] coordinates {
									(-33.75, 1)
									(-31.25, 1)
									(-28.75, 1)
									(-26.25, 3)
									(-23.75, 5)
									(-21.25, 7)
									(-18.75, 12)
									(-16.25, 14)
									(-13.75, 27)
									(-11.25, 40)
									(-8.75, 41)
									(-6.25, 56)
									(-3.75, 80)
									(-1.25, 108)
									(1.25, 114)
									(3.75, 127)
									(6.25, 111)
									(8.75, 110)
									(11.25, 79)
									(13.75, 72)
									(16.25, 48)
									(18.75, 37)
									(21.25, 23)
									(23.75, 20)
									(26.25, 12)
									(28.75, 13)
									(31.25, 3)
									(33.75, 6)
									(36.25, 3)
									(41.25, 1)
									(43.75, 2)
									(46.25, 1)
									(48.75, 1)
								};
						\end{axis}

						\footnotesize\node[align=left,anchor=west] at ([xshift=1cm]myaxis.east) {
							Number of centres = 1179\\
							Mean = 4.3\\
							Standard Deviation = 10.9
						};

					\end{tikzpicture}

					\caption{Variability in A-Level mathematics results – summer 2020 vs. summer 2021.}
					\caption*{\textit{\centering\scriptsize Data source: UK Government's Department for Education}}

				\end{figure}

				\vspace{0.5cm}
				
			\end{block}

			\vspace{0.5cm}

			\begin{block}{Engaging Advanced Concepts}

				\vspace{1.4cm}

				\begin{minipage}{0.28\textwidth}
					\centering
					\includegraphics[width=\linewidth]{./images/1-2-1-BrushOnCanvas.jpeg}
				\end{minipage}
				\hfill
				\begin{minipage}{0.7\textwidth}
					\textbf{Descriptive vs. Inferential Statistics}
					\begin{itemize}
						\item \textbf{Descriptive Statistics:} Brush strokes on a canvas - showing the immediate picture.
						\item \textbf{Inferential Statistics:} Predicting the next masterpiece - based on past works.
					\end{itemize}
				\end{minipage}

				\vspace{2.5cm}

				\begin{minipage}{0.28\textwidth}
					\centering
					\includegraphics[width=\linewidth]{./images/1-2-2-WhisperGame.jpeg} 
				\end{minipage}
				\hfill
				\begin{minipage}{0.7\textwidth}
					\textbf{Error Propagation}
					\begin{itemize}
						\item Like a game of telephone - small errors grow when combined.
						\item Understanding this - for decisions with accurate information.
					\end{itemize}
				\end{minipage}

				\vspace{2.5cm}

				\begin{minipage}{0.28\textwidth}
					\centering
					\includegraphics[width=\linewidth]{./images/1-2-3-Puzzles.jpeg}
				\end{minipage}
				\hfill
				\begin{minipage}{0.7\textwidth}
					\textbf{Matching Models with Reality}
					\begin{itemize}
						\item Models as puzzles - each piece a bit of data.
						\item Our job - put it together - accurately reflects the real world.
						\item Helps understand complex phenomena - through a simpler framework.
					\end{itemize}
				\end{minipage}

				\vspace{1.4cm}

			\end{block}

		\end{column}

		% Column 2 - centre - title block
		\begin{column}{.33\textwidth}

			\vspace{-0.82cm}

			\begin{block}{}
				\centering
				\Large \textbf{\inserttitle}\\

				\vspace{1cm}

				\normalsize \insertauthor\\

				\vspace{1cm}

				\small \textit{\insertinstitute}\\

				\vspace{1cm}

			\end{block}

			\vspace{0.5cm}

			\begin{block}{The Fascinating World of Statistics: Why It Matters \& How It's Used}

				Statistics is more than just numbers and graphs. It's the backbone of decision-making in our daily lives, science, and business.

				\vspace{0.5cm}

				Here's how statistics lights up the path in various fields, making the complex simple and the uncertain clear:

				\vspace{2cm}

				\begin{minipage}{0.28\textwidth}
					\centering
					\includegraphics[width=0.88\linewidth]{./images/2-2-1-StockMarket.jpeg}
				\end{minipage}
				\hfill
				\begin{minipage}{0.71\textwidth}
					\textbf{In Economics}
					\begin{itemize}
						\item \textbf{Stock Markets:} Use statistics to predict future trends.
						\item \textbf{Data Analysis:} Helps forecast market movements.
						\item \textbf{Informed Decisions:} Aids investors in making smarter choices.
					\end{itemize}
				\end{minipage}

				\vspace{2.5cm}

				\begin{minipage}{0.28\textwidth}
					\centering
					\includegraphics[width=0.88\linewidth]{./images/2-2-2-AirPollution.jpeg}
				\end{minipage}
				\hfill
				\begin{minipage}{0.71\textwidth}
					\textbf{In Environmental Science}
					\begin{itemize}
						\item \textbf{Air Quality:} Statistics reveal trends in pollution levels.
						\item \textbf{Time Tracking:} Offers insights over different periods.
						\item \textbf{Planet Protection:} Provides data to safeguard our environment.
					\end{itemize}
				\end{minipage}

				\vspace{2.5cm}

				\begin{minipage}{0.28\textwidth}
					\centering
					\includegraphics[width=0.88\linewidth]{./images/2-2-3-SocialMedia.jpeg}
				\end{minipage}
				\hfill
				\begin{minipage}{0.71\textwidth}
					\textbf{In Social Media Analysis}
					\begin{itemize}
						\item \textbf{Engagement Analysis:} Understands why some posts perform better.
						\item \textbf{User Data:} Analyzes clicks, shares, and viewing times.
						\item \textbf{Content Optimization:} Helps tailor posts for wider reach.
					\end{itemize}
				\end{minipage}
				
				\vspace{2cm}				

			\end{block}

			\vspace{0.5cm}
			
			\begin{block}{A World Without Statistics}
				
				\vspace{1.1cm}
				
				\begin{minipage}[t][0.3\textwidth][t]{0.3\textwidth}
					\centering
					\includegraphics[width=0.7\linewidth]{./images/2-3-1-PopulationOverload.jpeg}\\
					\vspace{0.5cm}  
					\textbf{Population Overload}\\
					\small We can't analyze or predict population trends, risking unsustainable growth.
				\end{minipage}
				\hfill
				\begin{minipage}[t][0.33\textwidth][t]{0.33\textwidth}
					\centering
					\includegraphics[width=0.64\linewidth]{./images/2-3-2-Inflation.jpeg}\\
					\vspace{0.5cm}  
					\textbf{Inflation Out of Control}\\
					\small Controlling inflation is guesswork. Imagine paying £1,000 for fish and chips!
				\end{minipage}
				\hfill
				\begin{minipage}[t][0.3\textwidth][t]{0.3\textwidth}
					\centering
					\includegraphics[width=0.7\linewidth]{./images/2-3-3-YourThoughts.jpeg}\\
					\vspace{0.5cm}
					\textbf{Your Thoughts?}\\
					\small What aspects of your daily life do you think would be impacted without statistics?
				\end{minipage}

				\vspace{2.92cm}
				
			\end{block}
			
		\end{column}

		% Column 3
		\begin{column}{.3\textwidth}

			\begin{block}{Making Statistics Fun with Data Visualization}

				\vspace{0.5cm}

				Let's dive into how sports teams, like \textbf{Ross Venus}'s ice hockey team, use statistics and data visualization to scout talent and predict game outcomes. By analyzing player performance data over the season, teams can create visual representations to compare players, predict future performance, and make strategic decisions.

				\vspace{0.5cm}

				It's not just about the goals. It's about understanding the player's journey and potential through the lens of data.

				\vspace{1cm}

				\begin{figure}[htbp]
					
					\centering
					
					\begin{minipage}[t][0.4\textwidth][t]{0.3\textwidth}
						\hspace*{5mm}
						\vspace{0cm}
						\includegraphics[width=0.9\linewidth]{./images/3-1-1-RossVenus}
						\caption{Ross Venus}
						\vspace{-0.3cm}
						\caption*{\textit{\centering\scriptsize Photo: Scott Wiggins}}
					\end{minipage}
					\hfill
					\begin{minipage}{0.65\textwidth}
						\vspace{-2cm}
						\begin{table}
							\centering
							\caption{Scoring Data for Ross Venus}
							{\footnotesize
								\begin{tabular}{|l|l|l|l|l|l|l|}
									\hline
									Season  & Team                    & League        & GP & G  & A  & TP \\ \hline
									2006-07 & Coventry Blaze U16      & England U16   & 7  & 0  & 3  & 3  \\
									2007-08 & Coventry Blaze U16      & England U16 2 & 9  & 9  & 16 & 25 \\
									2008-09 & Coventry Blaze U16      & England U16   & 14 & 14 & 16 & 30 \\
									2009-10 & Coventry Blaze U16      & England U16   & 15 & 42 & 16 & 58 \\
									2021-22 & Coventry Blaze          & EIHL          & 54 & 8  & 21 & 29 \\
									2022-23 & Coventry Blaze          & EIHL          & 53 & 9  & 23 & 32 \\
									2023-24 & Milton Keynes Lightning & NIHL          & 36 & 27 & 55 & 82 \\ \hline
								\end{tabular}
							}
						\end{table}
					\end{minipage}
					
				\end{figure}

				\vspace{-3cm}

				\begin{figure}[ht]
					
					\centering
					
					\begin{tikzpicture}
						
						\begin{axis}[
								title={},
								xlabel={Season},
								xlabel style={font=\footnotesize},
								ylabel={Scores},
								ylabel style={font=\footnotesize},
								width=30cm,
								height=11.2cm,
								symbolic x coords={2006-07,2007-08,2008-09,2009-10,2021-22,2022-23,2023-24},
								xtick=data,
								tick label style={font=\tiny},
								legend style={font=\tiny},
								legend pos=north west,
								ymajorgrids=true,
								grid style=dashed,
							]

							\addplot[mark=*,violet] coordinates {
									(2006-07,3)
									(2007-08,25)
									(2008-09,30)
									(2009-10,58)
									(2021-22,29)
									(2022-23,32)
									(2023-24,82)
								};

							\addlegendentry{Total Points}

							\addplot[mark=square*,cyan] coordinates {
									(2006-07,0)
									(2007-08,9)
									(2008-09,14)
									(2009-10,42)
									(2021-22,8)
									(2022-23,9)
									(2023-24,27)
								};

							\addlegendentry{Goals}

							\addplot[mark=triangle*,black] coordinates {
									(2006-07,3)
									(2007-08,16)
									(2008-09,16)
									(2009-10,16)
									(2021-22,21)
									(2022-23,23)
									(2023-24,55)
								};

							\addlegendentry{Assists}
						\end{axis}
						
					\end{tikzpicture}

					\caption{Scoring Trends of Ross Venus Across Different Seasons}
					\caption*{\textit{\centering\scriptsize Data source: Elite Prospects}}

				\end{figure}

				\vspace{0.5cm}

			\end{block}
			
			\vspace{0.5cm}
			
			\begin{block}{Data Visualization Journey}
				
				\vspace{2cm}
				\centering
				
				\begin{tikzpicture}
					
					\tikzstyle{icon} = [minimum width=5cm, minimum height=2cm, node distance=8cm]
					\tikzstyle{description} = [text width=13cm, align=center, node distance=1cm]
					\tikzstyle{arrow} = [thick,->,>=stealth]
					\tikzstyle{number} = [circle, fill=white, draw, font=\bfseries]
					
					\node[icon] (collecting) {\includegraphics[width=5cm]{./images/3-2-1-DataCollect.png}};
					\node[icon, right=of collecting] (analysis) {\includegraphics[width=5cm]{./images/3-2-2-DataAnalysis.png}};
					\node[icon, below=of collecting] (visualization) {\includegraphics[width=5cm]{./images/3-2-3-DataVisualize.png}};
					\node[icon, right=of visualization] (decision) {\includegraphics[width=5cm]{./images/3-2-4-DecisionMaking.png}};
					
					\node[description, below=of collecting] (desc1) {\textbf{Collecting Data}\\Gathering\\ performance metrics};
					\node[description, below=of analysis] (desc2) {\textbf{Analysis}\\Identifying trends and patterns};
					\node[description, below=of visualization] (desc3) {\textbf{Visualization}\\Creating visual representations};
					\node[description, below=of decision] (desc4) {\textbf{Decision Making}\\Informing\\ strategic decisions};
					
					\node[number] at ([xshift=-2cm]collecting.west) (num1) {1};
					\node[number] at ([xshift=-2cm]analysis.west) (num2) {2};
					\node[number] at ([xshift=-2cm]visualization.west) (num3) {3};
					\node[number] at ([xshift=-2cm]decision.west) (num4) {4};

				\end{tikzpicture}

				\vspace{1.8cm}
				
			\end{block}

		\end{column}
		
	\end{columns}
	
\end{frame}

\end{document}